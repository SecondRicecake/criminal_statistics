\documentclass[]{article}
\usepackage{lmodern}
\usepackage{amssymb,amsmath}
\usepackage{ifxetex,ifluatex}
\usepackage{fixltx2e} % provides \textsubscript
\ifnum 0\ifxetex 1\fi\ifluatex 1\fi=0 % if pdftex
  \usepackage[T1]{fontenc}
  \usepackage[utf8]{inputenc}
\else % if luatex or xelatex
  \ifxetex
    \usepackage{mathspec}
  \else
    \usepackage{fontspec}
  \fi
  \defaultfontfeatures{Ligatures=TeX,Scale=MatchLowercase}
\fi
% use upquote if available, for straight quotes in verbatim environments
\IfFileExists{upquote.sty}{\usepackage{upquote}}{}
% use microtype if available
\IfFileExists{microtype.sty}{%
\usepackage{microtype}
\UseMicrotypeSet[protrusion]{basicmath} % disable protrusion for tt fonts
}{}
\usepackage[margin=1in]{geometry}
\usepackage{hyperref}
\hypersetup{unicode=true,
            pdftitle={Assignment1024},
            pdfauthor={Suz},
            pdfborder={0 0 0},
            breaklinks=true}
\urlstyle{same}  % don't use monospace font for urls
\usepackage{color}
\usepackage{fancyvrb}
\newcommand{\VerbBar}{|}
\newcommand{\VERB}{\Verb[commandchars=\\\{\}]}
\DefineVerbatimEnvironment{Highlighting}{Verbatim}{commandchars=\\\{\}}
% Add ',fontsize=\small' for more characters per line
\usepackage{framed}
\definecolor{shadecolor}{RGB}{248,248,248}
\newenvironment{Shaded}{\begin{snugshade}}{\end{snugshade}}
\newcommand{\AlertTok}[1]{\textcolor[rgb]{0.94,0.16,0.16}{#1}}
\newcommand{\AnnotationTok}[1]{\textcolor[rgb]{0.56,0.35,0.01}{\textbf{\textit{#1}}}}
\newcommand{\AttributeTok}[1]{\textcolor[rgb]{0.77,0.63,0.00}{#1}}
\newcommand{\BaseNTok}[1]{\textcolor[rgb]{0.00,0.00,0.81}{#1}}
\newcommand{\BuiltInTok}[1]{#1}
\newcommand{\CharTok}[1]{\textcolor[rgb]{0.31,0.60,0.02}{#1}}
\newcommand{\CommentTok}[1]{\textcolor[rgb]{0.56,0.35,0.01}{\textit{#1}}}
\newcommand{\CommentVarTok}[1]{\textcolor[rgb]{0.56,0.35,0.01}{\textbf{\textit{#1}}}}
\newcommand{\ConstantTok}[1]{\textcolor[rgb]{0.00,0.00,0.00}{#1}}
\newcommand{\ControlFlowTok}[1]{\textcolor[rgb]{0.13,0.29,0.53}{\textbf{#1}}}
\newcommand{\DataTypeTok}[1]{\textcolor[rgb]{0.13,0.29,0.53}{#1}}
\newcommand{\DecValTok}[1]{\textcolor[rgb]{0.00,0.00,0.81}{#1}}
\newcommand{\DocumentationTok}[1]{\textcolor[rgb]{0.56,0.35,0.01}{\textbf{\textit{#1}}}}
\newcommand{\ErrorTok}[1]{\textcolor[rgb]{0.64,0.00,0.00}{\textbf{#1}}}
\newcommand{\ExtensionTok}[1]{#1}
\newcommand{\FloatTok}[1]{\textcolor[rgb]{0.00,0.00,0.81}{#1}}
\newcommand{\FunctionTok}[1]{\textcolor[rgb]{0.00,0.00,0.00}{#1}}
\newcommand{\ImportTok}[1]{#1}
\newcommand{\InformationTok}[1]{\textcolor[rgb]{0.56,0.35,0.01}{\textbf{\textit{#1}}}}
\newcommand{\KeywordTok}[1]{\textcolor[rgb]{0.13,0.29,0.53}{\textbf{#1}}}
\newcommand{\NormalTok}[1]{#1}
\newcommand{\OperatorTok}[1]{\textcolor[rgb]{0.81,0.36,0.00}{\textbf{#1}}}
\newcommand{\OtherTok}[1]{\textcolor[rgb]{0.56,0.35,0.01}{#1}}
\newcommand{\PreprocessorTok}[1]{\textcolor[rgb]{0.56,0.35,0.01}{\textit{#1}}}
\newcommand{\RegionMarkerTok}[1]{#1}
\newcommand{\SpecialCharTok}[1]{\textcolor[rgb]{0.00,0.00,0.00}{#1}}
\newcommand{\SpecialStringTok}[1]{\textcolor[rgb]{0.31,0.60,0.02}{#1}}
\newcommand{\StringTok}[1]{\textcolor[rgb]{0.31,0.60,0.02}{#1}}
\newcommand{\VariableTok}[1]{\textcolor[rgb]{0.00,0.00,0.00}{#1}}
\newcommand{\VerbatimStringTok}[1]{\textcolor[rgb]{0.31,0.60,0.02}{#1}}
\newcommand{\WarningTok}[1]{\textcolor[rgb]{0.56,0.35,0.01}{\textbf{\textit{#1}}}}
\usepackage{graphicx,grffile}
\makeatletter
\def\maxwidth{\ifdim\Gin@nat@width>\linewidth\linewidth\else\Gin@nat@width\fi}
\def\maxheight{\ifdim\Gin@nat@height>\textheight\textheight\else\Gin@nat@height\fi}
\makeatother
% Scale images if necessary, so that they will not overflow the page
% margins by default, and it is still possible to overwrite the defaults
% using explicit options in \includegraphics[width, height, ...]{}
\setkeys{Gin}{width=\maxwidth,height=\maxheight,keepaspectratio}
\IfFileExists{parskip.sty}{%
\usepackage{parskip}
}{% else
\setlength{\parindent}{0pt}
\setlength{\parskip}{6pt plus 2pt minus 1pt}
}
\setlength{\emergencystretch}{3em}  % prevent overfull lines
\providecommand{\tightlist}{%
  \setlength{\itemsep}{0pt}\setlength{\parskip}{0pt}}
\setcounter{secnumdepth}{0}
% Redefines (sub)paragraphs to behave more like sections
\ifx\paragraph\undefined\else
\let\oldparagraph\paragraph
\renewcommand{\paragraph}[1]{\oldparagraph{#1}\mbox{}}
\fi
\ifx\subparagraph\undefined\else
\let\oldsubparagraph\subparagraph
\renewcommand{\subparagraph}[1]{\oldsubparagraph{#1}\mbox{}}
\fi

%%% Use protect on footnotes to avoid problems with footnotes in titles
\let\rmarkdownfootnote\footnote%
\def\footnote{\protect\rmarkdownfootnote}

%%% Change title format to be more compact
\usepackage{titling}

% Create subtitle command for use in maketitle
\providecommand{\subtitle}[1]{
  \posttitle{
    \begin{center}\large#1\end{center}
    }
}

\setlength{\droptitle}{-2em}

  \title{Assignment1024}
    \pretitle{\vspace{\droptitle}\centering\huge}
  \posttitle{\par}
    \author{Suz}
    \preauthor{\centering\large\emph}
  \postauthor{\par}
      \predate{\centering\large\emph}
  \postdate{\par}
    \date{2019 10 24}


\begin{document}
\maketitle

\begin{enumerate}
\def\labelenumi{\arabic{enumi})}
\tightlist
\item
  Children's IQ scores are normally distributed with a mean of 100 and a
  standard deviation of 15. What proportion of children are expected to
  have an IQ between 80 and 120?
\end{enumerate}

mean=100 std=15 lowerbound=80 upperbound=120

\begin{Shaded}
\begin{Highlighting}[]
\CommentTok{#confidence interval}
\NormalTok{mean=}\DecValTok{100}
\NormalTok{std=}\DecValTok{15}

\KeywordTok{pnorm}\NormalTok{(}\DecValTok{120}\NormalTok{, mean, std, }\DataTypeTok{lower.tail=}\OtherTok{TRUE}\NormalTok{)}\OperatorTok{-}\StringTok{ }\KeywordTok{pnorm}\NormalTok{(}\DecValTok{80}\NormalTok{, mean, std, }\DataTypeTok{lower.tail=}\OtherTok{TRUE}\NormalTok{)}
\end{Highlighting}
\end{Shaded}

\begin{verbatim}
## [1] 0.8175776
\end{verbatim}

\begin{enumerate}
\def\labelenumi{\arabic{enumi})}
\setcounter{enumi}{1}
\tightlist
\item
  Generate data for hypothesis testing (student's T-test) and show me
  one exmaple each of a significant result and a non-significant result.
  Explain your thinking.
\end{enumerate}

\begin{Shaded}
\begin{Highlighting}[]
\NormalTok{raw_adhd<-}\KeywordTok{read.csv}\NormalTok{(}\DataTypeTok{file=}\StringTok{"adhd200.csv"}\NormalTok{)}
\NormalTok{df_adhd<-}\KeywordTok{data.frame}\NormalTok{(raw_adhd)}
\NormalTok{an_children<-df_adhd[df_adhd}\OperatorTok{$}\NormalTok{DX }\OperatorTok{!=}\StringTok{ "pending"}\NormalTok{,] }\CommentTok{#disregard pending diagnosis}
\NormalTok{girls<-an_children[an_children}\OperatorTok{$}\NormalTok{Gender}\OperatorTok{==}\DecValTok{0}\NormalTok{,]}
\NormalTok{girls_not<-girls[girls}\OperatorTok{$}\NormalTok{DX}\OperatorTok{==}\DecValTok{0}\NormalTok{,]}
\NormalTok{girls_diagnosed <-}\StringTok{ }\NormalTok{girls[girls}\OperatorTok{$}\NormalTok{DX}\OperatorTok{!=}\DecValTok{0}\NormalTok{,]}
\NormalTok{boys<-an_children[an_children}\OperatorTok{$}\NormalTok{Gender}\OperatorTok{==}\DecValTok{1}\NormalTok{,]}
\NormalTok{boys_not<-boys[boys}\OperatorTok{$}\NormalTok{DX}\OperatorTok{==}\DecValTok{0}\NormalTok{,]}
\NormalTok{boys_diagnosed <-}\StringTok{ }\NormalTok{boys[boys}\OperatorTok{$}\NormalTok{DX}\OperatorTok{!=}\DecValTok{0}\NormalTok{,]}


\CommentTok{#girls_diagnosed}
\CommentTok{#boys_diagnosed}

\NormalTok{boys_age <-boys_diagnosed}\OperatorTok{$}\NormalTok{Age}
\NormalTok{girls_age<-girls_diagnosed}\OperatorTok{$}\NormalTok{Age}
\end{Highlighting}
\end{Shaded}

T test for the age of girls and boys who were diagnosed:

\begin{Shaded}
\begin{Highlighting}[]
\NormalTok{res1<-}\KeywordTok{t.test}\NormalTok{(boys_age, girls_age, }\DataTypeTok{var.equal =} \OtherTok{TRUE}\NormalTok{)}
\NormalTok{res1}
\end{Highlighting}
\end{Shaded}

\begin{verbatim}
## 
##  Two Sample t-test
## 
## data:  boys_age and girls_age
## t = 2.0376, df = 75, p-value = 0.04511
## alternative hypothesis: true difference in means is not equal to 0
## 95 percent confidence interval:
##  0.03977795 3.51951617
## sample estimates:
## mean of x mean of y 
##  12.09200  10.31235
\end{verbatim}

Significance of comparing the age of girls and boys who were diagnosed:

\begin{Shaded}
\begin{Highlighting}[]
\NormalTok{res1}\OperatorTok{$}\NormalTok{p.value }\OperatorTok{<}\StringTok{ }\FloatTok{0.05}
\end{Highlighting}
\end{Shaded}

\begin{verbatim}
## [1] TRUE
\end{verbatim}

This implies that the difference between the mean of the two dataset can
be significant.

T test for the age boys who were diagnosed with adhd or boys who were
not:

\begin{Shaded}
\begin{Highlighting}[]
\NormalTok{res2<-}\KeywordTok{t.test}\NormalTok{(boys_not}\OperatorTok{$}\NormalTok{Age, boys_diagnosed}\OperatorTok{$}\NormalTok{Age, }\DataTypeTok{var.equal=}\OtherTok{TRUE}\NormalTok{)}
\NormalTok{res2}
\end{Highlighting}
\end{Shaded}

\begin{verbatim}
## 
##  Two Sample t-test
## 
## data:  boys_not$Age and boys_diagnosed$Age
## t = -1.5955, df = 104, p-value = 0.1136
## alternative hypothesis: true difference in means is not equal to 0
## 95 percent confidence interval:
##  -2.2654290  0.2453421
## sample estimates:
## mean of x mean of y 
##  11.08196  12.09200
\end{verbatim}

Significance of comparing the age of boys diagnosed with adhd, and
typical for age:

\begin{Shaded}
\begin{Highlighting}[]
\NormalTok{res2}\OperatorTok{$}\NormalTok{p.value }\OperatorTok{<}\FloatTok{0.05}
\end{Highlighting}
\end{Shaded}

\begin{verbatim}
## [1] FALSE
\end{verbatim}

This implies that the difference between the ages of boys who were
diagnosed and who were not is not siginificant, which is most likely the
result of using a dataset that was focusing on a certain age interval of
children.


\end{document}
